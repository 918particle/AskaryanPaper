\documentclass[12pt]{article}
\usepackage[margin=2cm]{geometry}

\begin{document}

The following report continues the discussion regarding manuscript DF12830/Hanson for Physical Review D.  May 12th, 2022 \\ \vspace{0.25cm}

\textit{We have included the report of the adjudicating referee C below.  We received this on April 15th, 2022.  We would like to thank the adjudicating referee C for the work, and we want to thank the editor for being willing to reach out to referee C.  The comments of referee C were helpful and encouraging.} \\

\textbf{General Comments about Progress with the Manuscript} \\ \\

I still believe that the referee report from Referee B was unprofessional.  While I continue to believe that, I have decided to edit the paper according to the changes suggested by Referee C. Before describing the significant changes we have made to the manuscript, following the suggestions made by Referee C, I would like to summarize once more why I think the report of Referee B was unprofessional:

\begin{itemize}
\item The general tone of Referee B's remarks did not align with the fact that they were unable to locate any issues with the physics.  There was no physics in the critiques of Referee B.  There were no critiques of the physics by Referee A, and Referee C acknowledges that Referee B could have phrased their report much better, and that it was far too harsh.  I am writing this paper as a member of an institution that serves historically under-represented students.  I wrote a similar paper on the same topic while serving at a large institution, and at no point did any referee raise such alarms about such minor issues as Referee B has done.  Though one can never ``prove'' discrimination, I felt the need to suggest to the editor earlier in this process that we were being asked to play by different rules than other institutions.  Actions like those taken by Referee B are exactly the kind that discourage new participation in professional science by first-generation students, and students of color.
\item There was no complete itemized list of proposed changes, but many repetitions and variations of the phrase ``This list of corrections is not complete ...'' In my career I have never seen a referee withold a complete list of proposed changes.  You cannot improve what someone does not tell you is wrong.  We would have appreciated an itemized list of suggestions that we could have addressed in turn.  Referee A provided that, and we responded as soon as we could.  Referee B did not take the time to create that, but did take the time to read and review other papers I have written.  Referee A stuck to the manuscript at hand, which is also what I do when I write referee reports.  That is the professional expectation.  We are not being asked to find any error by the author in past manuscripts, but the manuscript we are sent.  Referee B did not do that.
\item In the report from Referee B, Referee B claimed that one correlation coefficient was greater than another, but the actual data in the manuscript showed the exact opposite result.  At another point, Referee B was repeating the definitions of variables we defined, and wrote the wrong definitions (literally equations 1 and 2 of the manuscript).  In my humble opionion, that is further evidence of a lack of professionalism.
\end{itemize}

Report of Adjudicating Referee C, DF12830/Hanson, April 15th, 2022 \\ \vspace{0.25cm}
\hrulefill

\textbf{Preface} \\ 

I have been asked to act as an adjudicator of sorts between the authors and Referee B. The
original recommendation of Referee A was for some revision of a fundamentally interesting
manuscript, which the authors clearly responded to satisfactorily. Referee B finds it difficult
to determine the newness of the manuscript, and makes many critiques of the article. The
authors then consider that B’s comments were not made in good faith, and present
arguments accordingly, with – to my knowledge – no change in the manuscript.

To resolve what currently is an impasse between the authors and B, I consider the following
questions:

\begin{itemize}
\item What is the relevance of the article?
\item Are the methods valid?
\item Are the remaining critiques of B reasonable or in bad faith?
\end{itemize}

I then make some recommendations on how to proceed from here. Throughout, I do not comment on the precise details of the article or accuracy of specific critiques, but focus on the refereeing process itself. \\

\textbf{What is the relevance of the article?} \\

The stated aim of this manuscript, from page 2, is ``to present a fully analytic time-domain model valid for all viewing angles, provided that $\eta < 1$.''  Referee B however states: ``The theoretical results are basically the same as those presented in ref [31], only that they are now given in the time domain and the form factor is evaluated with simple approximations of the shower model, a gaussian for the depth development and an exponential decay for the lateral distribution.''  These two statements are mutually consistent– on page 2, the authors note that semi-analytical treatments exist which allow a given model of shower development – taken from Monte Carlo – to be produce an implied field strength.  A major criticism from Referee B therefore is that one could trivially take ref 31, insert one's own analytic treatment, and proceed – and I agree with Referee B that the theoretical aspects of this are not sufficient to warrant a new paper.

However, further down, referee B states ``It is true that simple models for radio pulses could be of much use at different levels in such experiments, as the authors state.''  Referee A acknowledges this in their remarks, and considers this relevant, stating: ``The manuscript derives a fully analytic model of Askaryan radiation in ice in the time domain. This manuscript will make an excellent addition to the literature on modeling Askaryan signals in ice.'' Thus when Referee B asks: ``perhaps the new contribution is to give an explicit expression in the time domain for the radio pulse away from the Cherenkov cone.''  I think the answer is clearly ``Yes.'' Therefore, I think all parties concerned – A, B, authors and myself – can agree that if the evidence is there that this paper presents an accurate analytical calculation of Askaryan pulses, it is worthwhile publishing.

\clearpage

\textbf{Validity of methods} \\

This brings in the question of physical accuracy – are the methods valid in reaching this conclusion?  On this, Referee B states: ``There is a phenomenological part of the article that compares these analytical results in the time domain to an alternative calculation. This, I believe, is new and maybe the authors are intending to justify the approximations made in the new analytical expression by making a detailed comparison of the result with a reference calculation.'' I.e. clearly an analytic treatment needs to prove its relevance by comparison to more accurate methods.  I do not see anything in B’s report that questions the accuracy of the comparison. Their comments, which begin: ``The authors discuss in much detail the behavior of the results obtained in the time domain, relating them to shower properties ...'' can be dealt with via small changes, and does not affect the publishability of the paper.  Furthermore, A considers this a worthwhile addition that addresses their original question on accuracy. Thus I conclude that, at its core, the article absolutely is publishable in PRD.  However, it does not mean that the article should necessarily be published in its current form. This is addressed in the next section. \\

\textbf{Are the remaining critiques of B reasonable or in bad faith?} \\

In their reply to B, the authors state: ``What we did receive from Referee B that is unprofessional ... '' This is a significant comment that should be addressed regardless of other concerns.  Going into particulars, the authors object that `` `Telling authors that major rewriting of the article is required for it to be published in any journal' comes across as an attempt to block the paper on any grounds one can find, and not something done in good faith.''  To assess this comment requires more context.  Referee B makes two broad categories of comments on the paper which do not affect its fundamental suitability for publication, but do imply major changes. These are on the topics of notation, and general length/writing style. It is in that context that B makes the above remark, concluding: \\

``There are many other issues of the paper that I am not willing to spend more time in writing
because the condition for me to accept this article for publication is that the authors should
rewrite it according to all these recommendations and concerns.'' \\

To which the authors respond: ``there are real questions about whether this report was done in good faith.'' and ``Although we welcome any additional responses from Referee B, we would like a more professional response.''

Let me first state that, given this history of the article, I somewhat sympathise with the emotions of the authors - they thought they had an article close to acceptance, then they are asked to perform a major revision.  However – I do not think the report of Referee B was made in bad faith. Without delving into details such as the relevance of the LPM effect for the article, B clearly does go into detail regarding their comments, and their recommendations regarding improving notation and shortening the article appear to be genuine suggestions to make the article a better contribution to science.  I also don’t think that the recommendations from B are contradicted by A - in my experience, many referees will not consider questions on the appropriate length of an article. Indeed, A suggests moving much of the material to the appendices, and while they are satisfied with the outcome, there is a clear shared sentiment.

Given that general agreement from all parties is that the mathematical derivations are not new, I am also surprised that the article is as long as it is. In that context, the general sentiment of B’s comments – to reduce the length of the article, and clarify precisely what is and is not new - is valid, although the authors should feel free to argue regarding the specifics. So to answer the question: no, I do not think that B’s report is made in bad faith. I think the comments are made in good faith with a genuine intent to improve the paper, and should be regarded as such. This is not to say that they are phrased as well as they might be - and indeed, are probably overly critical – but they should not be dismissed. \\

\textbf{Where from here?} \\

The authors have replied to B’s comments both in general and the specific. However, I think that their replies may have been coloured by their interpretation of B’s report being in bad faith. While they should feel free to argue for/against changes in the manuscript, I do not see any attempt to take B’s comments into consideration.  I do believe that some effort should be made along the following lines:

\begin{itemize}
\item Re-examine the manuscript in case there are instances where the clarity on what
us/is not new can be improved
\item Shorten the article to necessary information, keeping in mind the fundamental goal
is to provide and validate a fully analytic treatment, but not a new method of
deriving fields from cascade parameters
\item Simplify notation
\end{itemize}

Nonetheless, I also doubt that future dialogue between the authors and B would be fruitful.  I do not wish to suggest that I or anybody else act as a third referee in this instance.  My suggestion on how to proceed therefore is that the authors reconsider B’s comments.  After all, they are aimed to improve readability. I then suggest that the Editor checks that such an effort has been made, and makes a determination to publish accordingly. I suggest an editorial role here since readability and clarify will strongly affect the impact of the article (as measured by e.g. citations) and in this both the authors and the Editor have a mutual interest.  I do not suggest a further round of refereeing. \\

Second report of Referee A, DF12830/Hanson, December 6th, 2021 \\ \vspace{0.25cm}
\hrulefill

The authors have adequately addressed my concerns and I can see the paper is significantly improved with the reorganization and updates to the simulations. My only comment is the the signals in Fig. 6 (right panels) are clipped and the authors should consider expanding the y-axis. \\

\textit{We thank Referee A for all the advice and guidance with the manuscript.  We recongize how valuable your time is and we appreciate it.} \\ \vspace{0.25cm}

Report of Referee B, DF12830/Hanson, December 6th, 2021 \\ \vspace{0.25cm}
\hrulefill

\textit{We would like to provide some context about the manuscript that Referee B might not know.  This manuscript was first submitted on June 10th, 2021.  Referee A filed a report and we received it on July 29th, 2021.  In that initial report Referee A wrote the following about the topic and scope of the paper:} \\

\begin{quote}
The manuscript derives a fully analytic model of Askaryan radiation in ice in the time domain. \textbf{This manuscript will make an excellent addition to the literature} on modeling Askaryan signals in ice ...
\end{quote}

\textit{Referee A then presented a list of five areas where the manuscript could be improved.  These remarks involved stylistic and structural critiques.  None involved fundamental physics questions about any of the derivations or approximations.  One question was about the analysis approach in the second half of the paper.  Referee B refers to this section as the phenomenological section.  We performed this analysis again, with the modifications suggested by Referee A.  The results were improved slightly or unchanged, depending on the case.  By October 4th, 2021, we re-submitted our manuscript.  The editor noted that he was searching for a second referee, and by December 4th, 2021, we received a new referee report.  Referee A wrote the following (from above):}

\begin{quote}
\textbf{The authors have adequately addressed my concerns} and I can see the paper is significantly improved with the reorganization and updates to the simulations.
\end{quote}

\textit{We assumed that the report from Referee B would be of a similar nature: a concrete list of suggestions to improve the manuscript, especially six months into the process.  What we did receive from Referee B that is unprofessional.  Telling authors that ``major rewriting of the article is required for it to be published in any journal'' comes across as an attempt to block the paper on any grounds one can find, and not something done in good faith.  How can Referee B know if the manuscript would or would not be published in \textbf{any} journal?  Referee A asked us to re-write or re-organize sections in ways that run counter to the demands made by Referee B.  For example, Referee A specifically requested that we introduce the appendices:}

\begin{quote}
I recommend [appendices] to keep the main body of the text shorter, while maintaining the authors’ preference for presenting the complete derivation (which I appreciate).
\end{quote}

\textit{The original remark may be found below.  We have shared our concerns about this report with the editor, who agreed to bring in an adjudicating referee at the end of the process.  However, the editor has asked me to first submit a formal response.}

\textit{I have made an attempt to categorize the critiques from the report of Referee B and address them systematically.  First, there appear to be questions about whether the manuscript represents work that is of sufficient novelty to merit publication in PRD, including questions about distinguishing between review material and new material.  Second, there are questions about the meaning of citations.  Finally, there are questions about notation, variables, and the details of derivations.  There are also recurring themes, like the relationship between the mathematical and phenomenological parts of the paper, and the overall length.  To make ourselves clear when responding to the three areas, we should develop some keywords.}

\textit{In our responses below, we will use the phrase \textbf{``apparently not''} to respond to critiques in which our response involves invoking the outstanding match our analytical model achieves when it is compared to a model derived in a completely different way.  Semi-analytic models such as citations [19] and [30] (original manuscript) have both a Monte Carlo component and an analytic component.  Our work from 2017 in citation [20] (original manuscript) was derived in the Fourier domain.  We converted our model to the time-domain with great care and effort, attempting to preserve validity for any viewing angle.  This was no easy task, given the complexity of the original equations.  Correlation coefficients between our new time-domain model and the semi-analytic examples exceed 95\%, with fractional power differences of $\approx 5\%$.  That cannot be an accident, and we see that Referee B acknowledged that in the report.  While we do not have access to the ZHS Monte Carlo, the ARVZ model ([19] and [30] in original manuscript) matches our model, and the ARVZ model matches ZHS Monte Carlo.  Thus, our model makes the same prediction as the ZHS Monte Carlo.}

\textit{In our responses below, we will use the phrase ``\textbf{citation misinterpretation}'' to refer to the following problem.  Referee B is rightly concerned with properly citing sources.  When we provide a citation, we are not claiming credit for the implied discovery or finding.  We are providing one example for a skeptical reader who would not otherwise believe the claim or assumption we are making.  Suppose we write: ``The lateral ICD is distributed exponentially near the cascade maximum [ab].''  The paper we associate with citation [ab] does not always have to correspond to the earliest papers ever written containing this finding.  Sometimes, an example is appropriate.  If every claim had to cite all work back to the original finding, then the bibliography of every PRD paper should have hundreds of citations.  There is a reasonable limit to how far back we must cite others' work.  To address the concerns of Referee B, we have added citations and rearranged others.  (Note that the citation numbers have changed slightly).  We are also willing to include any citations Referee B suggests.  However, Referee B has raised this issue to an unprofessional level of gravity.} \\ 

\textbf{Remarks about Novelty, and distinctions about review material and new material} \\

\textit{To our knowledge, the literature on modeling the Askaryan effect does not yet contain a fully analytic model of the Askaryan field $\vec{E}(t,\theta)$ in the time-domain for $\theta_{\rm C}$ and $\theta \neq \theta_{\rm C}$.  In the introduction section, we place our model in the appropriate context by describing the development of full MC and semi-analytic work.  We articulated several advantages a fully analytic model would provide to radio detectors like IceCube Gen2 (radio) in the abstract and in part B of Section VII: Conclusion.  According to Referee A, and in our professional judgement, there is no question about whether this work is new enough to merit publication.  What follows is our response to individual comments made by Referee B about this topic.}

``However as one one gets down to the details and compares the results with those of earlier publications, the article starts rising many doubts, in particular about what is really new in the theoretical calculation presented, and about what are the authors actually checking when they compare their analytic expressions to the results obtained with another `semi-analytical' approximation.''

\textit{This model presents the details of the on-cone derivation ($\theta = \theta_{\rm C}$) of $\vec{E}(t)$ from citation [20] (of the original manuscript), and a derivation of the off-cone ($\theta \neq \theta_{\rm C}$) field $\vec{E}(t,\theta)$.  This work demonstrates the effect of the complex poles on the actual functions for $\vec{E}(t,\theta_{\rm C})$, and $\vec{E}(t,\theta)$ and the ensuing match with a completely different model.  Further, to our knowledge, no one has published an off-cone fully analytic model.  The effect of the poles is validated against the output of NuRadioMC based on citations [23] and [30] (of the original manuscript) and for both the on-cone and off-cone cases, and for electromagnetic and hadronic cascades.  We are completely open about how this comparison is done.  The authors of [19] and [30] (of the original manuscript) had access to the the ZHS MC code, but it is not public.  We cannot compare our model to ZHS MC.  However, the ARVZ models in [19] and [30] (of the original manuscript) match the full ZHS MC, and our model matches the ARVZ one.  We decided to include the derivation of the form factor $\widetilde{F}(\omega,\theta)$ from citation [20] (original manuscript) only as a matter of convenience for the skeptical reader who should not be forced to read the paper from 2017 just to accept the details of the current manuscript.  While Referee B had the time, patience, and curiosity to read the background papers, assuming the average reader will also read them is unwise.  Referee A is in agreement.  We want the reader to see the connection between the form factor, and the longitudinal development, with the final product.}

``As a result I have been drawn into comparing the results of this paper to earlier work in some detail to come to a reasonable conclusion with respect to what is the actual reach of this paper. I have found that the paper has many repetitions of earlier work and that it is extremely difficult to say what is truly new in the analytical calculation. The theoretical results are basically the same as those presented in ref [31], only that they are now given in the time domain and the form factor is evaluated with simple approximations of the shower model, a gaussian for the depth development and an exponential decay for the lateral distribution. Moreover the exercise of converting the frequency domain to the time domain is already the main content of ref [20], so perhaps the new contribution is to give an explicit expression in the time domain for the radio pulse away from the Cherenkov cone.''

\textit{In Section I, we specifically describe what is new in this manuscript, and place it in the appropriate context of the development of the sub-field.  In Section II: Units Definitions, and Conventions, we describe Eq. (5) as containing the five essential pieces required to build the model before taking the inverse Fourier transform.  The goal is to build an all-$\theta$, all-$t$ analytic model by converting Eq. (5) into the time-domain.  We credit citation [31] (original manuscript) with $\vec{\mathcal{E}}$, but strongly disagree that the results of this work are the same as reference [31] (original manuscript).  The authors of [31] (original manuscript) do not provide an all-$t$, all-$\theta$ analytic model, nor do they provide a form factor motivated by Monte Carlo studies or examine its effect on the field in the time-domain.  In fact, John Ralston himself was the one who first suggested we produce the form factor and use it to obtain a model like this.  The GEANT4 studies that justify our form factor were presented in citation [20] (original manuscript), and we complete the model in this paper by examining the effect of the form factor (and other factors) on the analytic form of $\vec{E}(t,\theta)$.}

\textit{Moving on, Section III A provides the form factor to be entered into Eq. (5).  Section III B provides the reader with an understanding of the longitudinal length parameter, which is necessary to evaluate $\vec{\mathcal{E}}$.  We discuss below the remarks Referee B makes about Section III B in particular.  In Sections IV and V, we assemble for the first time the pieces for Eq. (5) and transform the model into the time-domain.  We make necessary approximations to keep the model analytic.  Section VI is also new, since no other paper to our knowledge performs this type of comparison (fully analytic to semi-analytic).  Some works in this genre compare the semi-analytic models to the ZHS MC (citations [19] and [30] of the original manuscript and ARZ, 2011 are good examples).}

\textit{In Section VII: Conclusion, part A, we summarize new results in Table IX.  In the text, we explain why we think each result in this table represents something new.  We explain that the form factor was first described in [20], but how its effect on the analytic form of $\vec{E}(t,\theta)$ is now presented in this work.  We explain how Ralston and Buniy introduced the length parameter $a$, but how we have finally connected it to the analytic form of $\vec{E}(t,\theta)$. We have also presented a novel approach for energy reconstruction.  In part B of Section VII, we justify our approach by listing four different cases in which a fully analytic time-domain model provides advantages to detectors like IceCube-Gen2.  At the end of the quote above, Referee B even acknowledges that ``perhaps'' the new contribution is the explicit expression for $\vec{E}(t,\theta)$.  Since we've demonstrated that what we're doing is new throughout the paper, the claim of ``nothing new'' is unjustified.}

``[In ref [20] the explicit expression of the pulse in the time domain is only given in the Cherenkov cone. There is a small difference between the two expressions but no comment is made on this difference.  Is this a correction? Moreover reference [20] treats LPM showers incorrectly, because the frequency spectrum of the pulses has the wrong low frequency limit (it should be the same as that of a shower of the same energy and no LPM effect). In this article the authors should discuss differences wrt ref [20] and address errors as well.]''

\textit{The above remarks come after the prior quote, and are related to work done in citation [20] (original manuscript).  The analytic form of $\vec{E}(t,\theta)$ was not the focus of citation [20] (original manuscript).  Citation [20] (original manuscript) does not address the uncertainty principle for on-cone signals, nor does it explain how the poles that lead to its shape connect to \textbf{both} the lateral ICD \textbf{and} longitudinal development.  Referee B states that the treatment of low frequency modes is wrong in [20] (original manuscript), when both reviewers of [20] (original manuscript) approved the analysis.  \textbf{Referee B states that the modes should match a cascade with no LPM effect.  We presented results in [20] (original manuscript) that demonstrate that they do, so this criticism is invalid.  Not only is it invalid, it is not even about this manuscript.}  Referee B should have understood that we avoided the LPM effect in this analytic model by using cascade energies of 10 PeV for EM cascades and 100 PeV for hadronic ones.  We state that explicitly in Section I.  Finally, does our model disagree with others' models that do include the LPM effect? \textbf{Apparently not.}  Please consult Section VI to observe the excellent agreement between a semi-analytic model that would have exhibited the LPM effect if it were relevant at these energies in ice, which it is not.}

``Leaving aside the inadequacy of the presentation of the results (to be addressed below), the immediate question that crops up is, is this result of enough new content to justify a new article in PRD? The answer in my opinion is that the result could be of interest for the field if the presentation of this is clear and the calculation leads to a compact expression (it appears to be so) which describes the pulses in this region with sufficient accuracy.''

\textit{If Referee B is acknowledging that a compact expression of an all-$\theta$ all-$t$ model is of interest for the field - ``it appears to be so'' - then why is Referee B raising concerns about novelty at all in this report?  If my colleagues had not demonstrated sufficient interest level, we would not have embarked on a six-month effort to get these findings published.}

``There is a phenomenological part of the article that compares these analytical results in the time domain to an alternative calculation. This comparison is used to find the fit parameters for the physical scales of the showers. This, I believe, is new and maybe the authors are intending to justify the approximations made in the new analytical expression by making a detailed comparison of the result with a reference calculation. Putting aside all the doubts that one can have about the procedure used for such a ``validation,'' the same question comes up, is this sufficient to be addressed in a PRD paper? In my opinion the answer is that maybe it is enough, given the outstanding effort that is being made to exploit the radio technique in ice in the last decade, and the current interest that the technique has for neutrino astronomy. It is true that simple models for radio pulses could be of much use at different levels in such experiments, as the authors state.''

\textit{This comparison is brand new in the literature.  To our knowledge, no one has produced a fully analytic all-$t$ all-$\theta$ model that matches a semi-analytic/Monte Carlo model this well.  Indeed, the match is of high quality, as quantified by correlation coefficients and fractional power differences.  Cross-correlation and fractional power differences are standard techniques used to establish a match between waveforms.  Further, one obtains $a$-values, $\omega_0$-values, $\omega_C$-values, and values for $\epsilon$ that all make sense in light of Section III.  Referee B acknowledges that we state reasons for why such a validated model would be of use to the community given the outstanding effort underway to observe neutrinos via the Askaryan effect.  Unless our stated justifications are wrong, why is there any question at all about the novelty and merit of this work?}

``There is a lot of spurious text. Parts of the text are clearly to be completely removed. I was particularly surprised to find out that section III B, that takes nearly two pages and seems an elaborate calculation, is simply an estimate of the width of the well known Greisen and Gaisser parameterizations, as obtained approximating them with gaussian functions. This is efficiently described in ref [31] already (which is the basis of this work) with a reference to Bruno Rossi. Most people working with showers have approximated the shower development with gaussians so it seems too trivial to be reported in detail. Moreover the derivation is particularly cumbersome. I do not even think it should be put in an appendix, the final result can just be displayed after a brief and clear description of what is actually being done. Incidentally references could be made to [31] and to the work of Andringa,S. et al. Astrop Phys 34 (2011) 360 describing both the gaussian approximation, fluctuations in the longitudinal spread (as named in [31], this is a better choice than ''longitudinal length'') and corrections to it. Incidentally the authors have payed very little attention to fluctuations in the shower profile which are relevant. It is particularly relevant to understand how the fluctuations in the longitudinal spread relate numerically to the small changes as the energy is varied. The sections on the uncertainty principle are in my opinion a distraction, it is not clear why they are introduced (nor is it clear that the principle is always satisfied, which is suspicious).''

\textit{Section III B requires 1.75 pages out of a 23 page manuscript - 7.5\% of the work.  The authors of citation [31] give the energy-dependence of the longitudinal length, but do not show the derivation.  Experimentalists and NuRadioMC developers in our field would never accept a simple formula for the longitudinal length without proof since they are not all cascade theorists, so we provided it for convenience.  Our derivation includes also the dependence of the $a-value$ on the fraction $f$ of $n_{max}$.  This is to aid with comparisons to the $a-values$ found in Section VI.  We have added the citation that Referee B suggests, Andringa et al (2011), and a few more for good measure.  However, the standard Referee B uses to remove content is illogical.  Referee B raises ``alarms'' about citations motivated by the concern that they are misleading to a non-expert reader.  However, Referee B also advocates cutting content because the average reader is an expert in cascade physics.  \textbf{The reader cannot be an expert and a non-expert.}  As we stated earlier, these remarks just come across as an attempt to find any reason to block the paper.  The derivation proves useful, for at the end of Section III B we calculate a typical $a$-value of $\approx 4$ meters for a 10 PeV event.  The theoretical result matches what we find when fitting our model to a completely different model.  Experimentalists in our field need proof that cascade theory matches our results, and we provide it.}

\textit{The LPM effect is irrelevant for the energies considered in this manuscript.  Do our models demonstrate disagreement with semi-analytic calculations that include the LPM effect?  \textbf{Apparently not.}  The implementation in NuRadioMC incorporates shower libraries that have the LPM effect activated, and we have validated our model against this NuRadioMC output.  NuRadioMC even produced a hadronic cascade with a sub-shower (waveform 5 in Fig. 10, and entry 5 in Table VIII).  Our model still shows a correlation coefficient of 95.5\% and a power difference of 8.76\%, at 100 PeV.  Why does Referee B require a discussion of how our model does nothing to account for the LPM effect when it isn't relevant to the analysis?  We state at the end of Section I that we are considering energies for which the LPM effect has negligible influence.  These energies represent the energy threshold region for IceCube Gen2 designs.  All models of UHE-$\nu$ predict a flux that increases with decreasing energy.  Thus, the threshold energies correspond to the largest number of UHE-$\nu$ and should be treated as important.}

\textit{The sections on the uncertainty principle are a mandatory check that we have done our transformation from the Fourier domain to the time domain correctly.  If an approximation we made affected the shape of the waveform incorrectly, the uncertainty principle would not hold.  In the off-cone case, we use the width of the field in the Fourier domain from citation [20] (original manuscript), and multiply by the time-domain width found in this manuscript.  This product is greater than or equal to $1/(2\pi)$ for $\eta \to 0$, a clear validation.  Referee B does not even specify how this result is not always satisfied.  Nor does Referee B specify what is wrong with the on-cone uncertainty principle check.  In that case, we find Eq. 49, which requires a sum of two ratios to be greater than $\Delta\theta$, which approaches zero.  Why is that controversial?  In the on-cone case, $\Delta\theta \to 0$, but $a/r$ and $l/a$ are both small and non-zero.} \\

\textbf{Citation Misinterpretation} \\

\textit{We discuss here the ``problem'' Referee B raises with respect to citations.  The first such remarks from Referee B begin here:}

``The authors discuss in much detail the behavior of the results obtained in the time domain, relating them to shower properties. A non-expert reader could be misled into thinking these are new findings, but most of them are actually well documented in the abundant literature on this topic starting with the Monte Carlo simulation of ref [13] and practically in all the articles that have followed up to date. Stating these properties is not a problem of course, moreover it is a plus because they serve as a check giving confidence that the calculation gives reasonable results, however proper references should be made to articles that have actually found the same properties with similar or different techniques.''

\textit{Of course we are willing to add citations Referee B suggests, but remarkably, Referee B does not provide a list of citations in the paper that are apparently causing such a problem.  We will add citations to the manuscript that Referee B feels are necessary (for example, we added Andringa, 2011).  But how can we respond to a referee report when it does not contain a complete list of revisions to be made?  The bulk of the discussion from Referee B begins here:}

``More careful referencing is compulsory. The article has many issues with the references which are not acceptable. These are actually the most striking alarms on a first read. The first one is part of the first general comment, the way the article references earlier work on theoretical calculations is misleading, because the authors are not clear enough about what is new and what is not. The article should make a fair summary of what is being done in relation to what was done in the past, particularly in ref [31], what ref [20] added to it and finally limit the first part of this article to the minimal amount of material that is needed to present the time expression that follows from the work of [20] and the approximations made for the form factor.''

\textit{Referee B is asking for a ``fair summary of what is being done in relation to what was done in the past.''  Did Referee B not read the final three paragraphs of Section I?  We dedicated half of a page to giving credit to examples of past work.  We specifically describe the thread of contributions by the authors of [13], [19], [20], and [31] (citation numbers from original manuscript).  Please read Section I, and make concrete suggestions about which citations are missing.  Note that we added Andringa, 2011, and others.}

``I will mention a few others but the list is not complete.  In several places an inadequate reference is given, for instance the authors cite Hanson et al [20] for the Greisen model, they cite again [20] for showers with many maxima because of the LPM effect but the showers described in [20] have no multiple peaks, and again refer to [20] when they say that the lateral cascade width has the effect of a ``low-pass filter,'' while [20] may be the first use of such a ``term'', such a reference is somewhat misleading in the sense that the effect of the cutoff in the frequency spectrum related to the lateral spread of the shower is described in multiple works on this topic, including the early ZHS MC calculation.''

\textit{Referee B has taken the time to read and cross-reference the background literature, but does NOT take the time to provide \textbf{a complete and exact list of proposed changes.}  That's a red flag, and comes across as an attempt to simply delay or block the manuscript for any reason that may be found.  Remarkably, the first mention of the Greisen distribution in Section II does NOT trigger a critique from Referee B, even though the citation is for the ARA collaboration and the associated Monte Carlo code, AraSim.  The reader is smart enough to know that we are citing the use of the Greisen distribution within AraSim, and not that the ARA collaboration discovered this distribution.  Referee B states that we cite [20] (from the original manuscript) for showers with multiple maxima, but does not say where.  Referee B could mean the third paragraph of Section 1.  Assuming that's the case, we've changed the citations there to refer to Gerhardt and Klein (2010), and Alvarez-Mu\~{n}iz, Protheroe, and Zas (2009).  Our original intent was to cite the study of LPM elongation in Hanson and Connolly (2017).}

\textit{Referee B could also have meant, for example, the last sentence of the first paragraph of Section VI, but that would be \textbf{citation misinterpretation.}  We are merely citing the fact that the LPM changes the shape of $n(z')$, and that [20] (from the original manuscript) accounted for LPM elongation, not multiple sub-showers.  Similarly, in the last paragraph of part A of Section VII, we remark that the longitudinal length, $a$ grows faster than $\sqrt{\ln(E_{\rm C}/E_{\rm crit})}$, and cite [20] (from the original manuscript) as an example.  If the remark of Referee B about the LPM effect referes to this example, that would again be \textbf{citation misinterpretation.}  In that instance, however, we removed the citation because it is redundant, but added a citation to Gerhardt and Klein (2010) as we did above.  Gerhardt and Klein (2010) provided the energy dependence of the longitudinal cascade development over the widest energy range we could find, and their work was helpful when we wrote our paper in 2017 (Hanson and Connolly, 2017).}

\textit{It is true that citation [20] in the original manuscript (Hanson and Connolly, 2017) found that the effect of the form factor as a low-pass filter with physical poles.  Stating that we are claiming original credit there is \textbf{citation misinterpretation}.  Did Referee B not encounter the sentences following Eq. 11 where we show how our $\widetilde{F}$ compares to the original form factor that ZHS gleaned from their Monte Carlo simulations?  Just to satisfy Referee B, we have added an additional ZHS reference, including in Section III A just before the remarks about Moli\`{e}re radius.}

``The citation of the semi-analytic approach, ref [19], states that the method was ``introduced ... to account for non-Gaussian fluctuations in the charge excess profile.''  This is simply not true, the method was introduced to provide a fast and accurate method to describe radio pulses in general (similar to the motivation behind the analytic calculation the authors are discussing). The list again is longer. Such a careless citation style is not acceptable.''

\textit{Referee B has a point that citation [19] (in the original manuscript) was not created just to account for non-Gaussian fluctuations in the charge excess.  It was created for a variety of reasons, one being that it accounts for a charge excess profile $n(z')$ with non-Gaussian fluctuations.  We have clarified the language in that paragraph.  Referee B repeats the mantra that ``the list again is longer,'' \textbf{but withholds the complete list}.  It would have been easy to give a list of citations to substitute or add.} \\ \\

\textbf{Questions about notation, variables, and derivations} \\ \\

\textit{The issue of notation and variable names should not be considered more than ``a minor issue about style.'' Referee A made specific suggestions and properly categorized them as issues of style.  We included those changes, and now Referee A has agreed to move forward.  Referee B begins commentary here:}

``A large number of variables is used together with several unfortunate choices of notation. Although I would usually classify such a comment under "minor issues about style", this article is particularly exaggerated in this respect and more importantly it becomes an obstacle to clarity, obscuring the reasoning and the expressions presented and contributing to the confusion of the reader. Here are some examples, again the list is not exhaustive: The authors use unnecessary changes of variables introducing names such as $p,q,u, x, x_{c}, \omega_{0}, \omega_{CF}, \omega_1$ ... that either giving little intuition to interpret the expressions, or I cannot see the need to introduce them all, and that in my opinion obscure the interpretation.''

\textit{Again we find the phrase ``the list is not exhaustive'' and reply: how can we respond if you make the criticism but withhold the list?  Let's start with what Referee B does specify: $p$.  We provide Tables I-IV with explicit definitions of variables and functions used in the derivation.  The variable $p$ first appears in the off-cone calculation in Table II, and later near Equation 55 (in the current draft) and in the definition of $\omega_1$.  The skeptical reader will follow Tables II-IV and Appendix C when directed to see how $p$ is introduced.  Other readers will just skip to the final result, Equation 57 (of the current draft) and use Table II.  We do not understand remarks like ``little intuition to interpret the expressions,'' or that these variables are not necessary.  The pulse width is $\sqrt{2p}$, so it is necessary and has a physical interpretation.}

\textit{Next, we have $q$, which fist appears in Sections II and III to remain faithful to the original RB definition of the form factor.  Once $\widetilde{F}$ is obtained, $q$ as $nk(1,\vec{\rho}/R)$ is dropped.  We use $q$ again in Section III B to simplify our argument that the longitudinal length can be predicted to match the results of Section VI.  The derivations of $a$-values for EM and hadronic cascades allows the readers to see for themselves exactly what $a$-value is obtained for a given energy and $f$, the fraction of $n_{max}$.  That can't be a bad thing.  Pages later in Section V, we reuse $q$ to simplify the RB field equations, and give its definition in Table II.  This type of variable is helpful because $\mathcal{E} = f \cdot g \cdot (1 - h)$, taking a very complex function and converting it to a simple form for Taylor expansion.  The same is true for $u$, $x$, and $x_{\rm C}$.  The definitions of these are given in Table II and used to simplify $\mathcal{E}$ for Taylor expansion.  The Taylor expansion in Eqs. 52 and 53 (current manuscript) are a critical step, and we think the reader should be able to verify it.}

\textit{Criticising the usage of $\omega_{0}$ and $\omega_{CF}$ is bizarre.  They come from JCH+AC, 2017 (citation [20] of original manuscript), making it necessary to say what they are.  They are also both necessary to see that the form factor follows the two-pole low-pass filter equation for $\sigma < 1$.  One could complain that only $\omega_{CF}$ is necessary, but it is not the actual location of the poles, which are at $\pm i \omega_{0}$.  The variable $\omega_1$ is necessary to perform the final steps to obtain $\vec{E}(t,\theta)$.}

``In places the authors introduce redundant variables, for instance there is a cutoff frequency $\omega_{CF}$ and a second one $\omega_0$ which is simply $\omega_0=sqrt(2/3) \omega_{CF}$. One could question the need to introduce two frequency scales which are so close, when a single one would do the job. The introduction of $\sigma = \omega/\omega_{CF}$ is another unnecessary new variable.''

\textit{The reason $\omega_{\rm 0}$ is necessary is given above.  The definition of $\sigma$ is not $\omega/\omega_{\rm CF}$.  The definition is given in Eq. A6: $\sigma$ is the ratio $k\sin\theta$ to $\sqrt{2 \pi}\rho_{\rm 0}$.  Referee B could have seen this by reading Appendix A of this manuscript, or in Hanson and Connolly, 2017.  The variable $\sigma$ is a ratio of the lateral component of the wavenumber to the inverse of the lateral exponential decay length of the ICD.  Said another way, it is proportional to the ratio of the decay length of the lateral ICD to the wavelength.  Thus, it does have intuitive physical meaning: the form factor filters wavelengths shorter than the lateral ICD width.}

``For the lateral distribution a simple exponential is used as an approximation. The radius is labeled ``rho'' in cylindrical coordinates and a length scale is introduced as $1/\rho_0/\sqrt{2 \pi}$ which is extremely confusing because here $\rho_0$ is used as an inverse radial distance. $1/\rho_0$ is related linearly to the Moliere radius (incidentally the definition given of the Moliere radius is incorrect). It is unnecessary to introduce two scales for the lateral spread given the simple approach given in this article, one of the two is enough to set the scale and using two adds to the confusion of variables, particularly with the unnatural definition of $\rho_0$.''

\textit{There are at least four reasons the reader will not be confused by the units of $\rho_{0}$.  a) We state explicitly that $\rho_0$ has units of inverse length in the sentences after Eq. 7.  b) Eq. 7 shows $\rho_0^2$ as the ICD normalization, and we state the units to be a number density.  The only way that the units work out is if $\rho_0$ has inverse length. c) In the section about Moli\`{e}re radius, in the first sentence, we discuss how $l = 1/(\sqrt{2\pi} \rho_0)$ is a length. d) All of Appendix A.  We took that definition because it helps make the form factor $\widetilde{F}$ a 2-pole low-pass filter response for $\sigma < 1$.  In Hanson and Connolly, 2017, we shared graphs of $1/(\sqrt{2\pi}\rho_0)$ in units of distance.}

``Another example of the latter type is introducing $u=1-i \eta$, which is not justified because afterwards an expansion is made in powers of $(1-u)=i \eta$, when $u$ tends to 1. Clearly $u$ is redundant and the natural variable to use is $\eta$, the limit corresponding to $\eta$ going to zero.''

\textit{Really?  Is it not enough that we just found it easier to re-organize the RB field equation (for the $\hat{\theta}$ component to be $\mathcal{E}(u,x) = f(u,x)g(u,x)(1-h(u,x))$, and then to expand around $(u - 1)$?  Equations C3 and C4 show how it's much easier to evaluate all the derivatives from the chain rule if the derivatives are with respect to $u$.  Besides, we switch back to $\eta$ in Eq. C7, so the final results do not even involve $u$.}

``These problems include a few examples of bad choices of variables and labels, but the list is longer. The article should reduce the number of variables, change names where they lead to confusion such as the previous case, and eliminate spurious definitions that obscure the mathematical steps. Incidentally what I am demanding about improving the notation would end with a notation very close to that used in reference [31], which is an excellent article and has a rational labeling of variables. It should be used by the authors as a guide of style. Moreover, using the same conventions would make it a lot simpler to relate their work to the original work by Buniy and Ralston.''

\textit{We encounter again a comment like ``the list is longer.''  By withholding a complete and exact list of changes, we are forced to guess what additional changes Referee B would like.  For the critiques that have been provided, we have shown that the steps taken are necessary and carry physical intuition.}

``The expression of the pulse in the time domain away from the Cherenkov angle could be of interest. Yet the presentation of this result is again not clear. The authors have split the calculation into the text and the appendix in a way that is not adequate because the text cannot be followed without reading the appendix. The expression involves several approximations, firstly an expansion for small $\eta=\omega/\omega_{CF}$ (more clear than in $(1-u)$). The result is an integral which is claimed to be related to the line-broadening function. The comparison involves the definition of $\omega_1=t_r/(2p)$, which depends on the retarded time, $t_r$. this is a variable and therefore it is not a number. So the condition for approximating the integral obtained by the symmetric approximation $\omega_1<\omega$ is clearly not satisfied for $p$ small, for $t_r$ large and obviously for low values of $\omega$ that are unavoidable in the $\omega$ integral from infinity to -infinity. The justification of this approximation is non-existent and deserves at least a careful discussion.''

\textit{1) Referee A specifically requested the appendices.  2) Check the DLMF for the definition of the line-broadening function.  3) The retarted time is treated as a constant in an integral over frequency.  4) If $p$ is small, that takes one to $\theta = \theta_{\rm C}$ (see Table II).  This is the \textbf{off-cone} case, so $p$ is not small.  We even discussed in Section V B when specifically we consider $\theta$ to be too close to $\theta_{\rm C}$.  5) For large $|t_{\rm r}|$ and $\omega \to 0$, the field amplitude rapidly approaches zero.  So does the approximation lead to a mismatch with the semi-analytic model?  \textbf{Apparently not.}  We added a remark that this is the case. We have also simplified Equation 56 (current manuscript) to be as clear as possible about the symmetric approximation, which is not a strong requirement and leads to good agreement between our model and another that has been further verified.  6) \textbf{Referee B writes that $\eta = \omega/\omega_{\rm CF}$, and that is not true.  We define $\eta = k(a \sin\theta)^2/r$, and we define $\eta = \omega/\omega_{\rm C}$.  These definitions are Equations 1 and 2 of the manuscript.  Does Referee B not understand the difference between $\omega_{\rm CF}$ and $\omega_{\rm C}$?  We don't understand the importance of changing the Taylor expansion variable, and we don't understand why Referee B is using an alternative definition for $\eta$.}}

``I am doubtful that this is a good approximation. In any case the authors could argue that the expression obtained this way is a reasonable approximation to the pulse. In this respect the latter part of the article, which compares these results to a reference calculation, takes some relevance. I did not find such reasoning though and if this is the intention of the authors it should be clearly stated. Indeed when one looks at the comparisons made for the off-cone case it is apparent that the agreement is a lot worse than for the on-cone case. Given the number of approximations made, another question crops up. What are the advantages of this expression relative to those of ref [13] that are intended to provide a good and rapid description of the pulses? This should be addressed for completitude.  Moreover the approach used is clearly not valid for the LPM effect (this effect was also addressed in ref [20] with flaws), while the approach of [13] can deal with fluctuations of the shower profile and with the completely different profiles of showers fully affected by the LPM effect. This issue deserves a discussion.''

\textit{With correlation coefficients greater than 95\% in every single case, you cannot claim that the agreement between the analytic and semi-analytic models is ``a lot worse.''  The on-cone, off-cone, electromagnetic, and hadronic cases all demonstrate fantastic agreement.  The correlation and $(\Delta E)^2$ results in Table VII for EM events are 0.989 and 2.2\%, and 0.973 and 5\% for Table VIII (hadronic events).  How do these compare to the on-cone results in Tables V and VI?  The on-cone EM results are 0.966 and 7.7\%, while the on-cone hadronic results are 0.99 and 1.9\%.  So not only is the remark from Referee B unjustified by the average results, the remark is \textbf{false} for the EM case: the off-cone EM fits are \textbf{better than} the on-cone ones.  As for the advantages of a fully analytic model relative to full MC and semi-analytic, we have given at least four in Section VII B.  Referee B even acknowledges that in some of the above remarks.  We actually give three reasons before the introduction in the abstract.  For the LPM effect, we do not know why Referee B did not find the language in the paper about how the LPM effect is not relevant for the cascade energies we considered.  The comparisons in Section VI would have been bad if the LPM effect mattered, but \textbf{apparently it does not.}}

``Incidentally a comparison is not made relative to a detailed simulation but rather to a calculation based on [13] that uses the simulation of the radio pulse in the Cherenkov cone to calculate the radio pulse away from it. This presumably has been done because [13] is fast but it is nevertheless an indirect way to compare. Some discussion of this is missing and multiple references are given to a simulation package that includes the approach of [13], what again leads to confusion because the reader may doubt whether they are comparing the parameterization to a MC simulation or to another model. If this is the case it needs to be discussed.''

\textit{The real reason is that we do not have access to the ZHS MC, but we do have access to NuRadioMC.  The NuRadioMC package is cited in our manuscript and it is what is being used to design IceCube Gen2.  We have performed two actions to address this remark.  First, we added a citation to the ARVZ (2020) model alongside a NuRadioMC citation in the first paragraph of Section VI.  Second, we added language to the first paragraph of Section VI beginning with ``To be clear ...'' to ensure the reader knows more precisely the origin of the semi-analytic waveforms.}

\textit{\textbf{This concludes our responses to the remarks of referee report from Referee B}.  Although we are grateful that Referee B dedicated the time to compare our manuscript to the background literature, there are real questions about whether this report was done in good faith.  In many instances, Referee B states that they are not providing the complete list of proposed changes, and that makes it difficult to guess what places in the manuscript need revision.  Some remarks from Referee B are not even supported by the results given.  The clearest example of this was the remark about the off-cone fit results of Section VI being ``a lot worse'' than the on-cone ones, when the EM data shows the opposite is true.  Another example is how Referee B repeatedly invokes the LPM effect as a missing discussion, when the LPM effect is not relevant for the energy range considered.  We state that clearly several times in the paper, and Section VI confirms good matches between models that have the LPM activated and our analytic one.  Finally, Referee B made a remark about notation that does not account for Equations 1 and 2 of the paper.  Although we welcome any additional responses from Referee B, we would like a more professional response.} \\ \\

\underline{Original Remarks of Referee B}

General discussion about suitability of the addressed topic: \\ 

This article is difficult to evaluate because it appears to be a thorough analytical calculation of radio pulses in ice in the time domain which seems to put in perspective and build from a couple of earlier articles on this topic, namely references [20] (involving one of the authors) and the original one [31] on the same subject (from 2001), and it provides expressions for the pulses in terms of parameters that relate to the geometry and to physical scales such as the longitudinal and lateral spread of the shower. In its second part the article seems to validate these results by a detailed comparison with an alternative calculation based on ref [19], which is used to fit the pulses to those obtained in the alternative approach and to get the best fit parameters. As such it could be both classified as a theoretical paper on the mathematical modelling of radio pulses in ice and also as a phenomenological article in which a given analytic parameterization is compared to a more precise calculation. However as one one gets down to the details and compares the results with those of earlier publications, the article starts rising many doubts, in particular about what is really new in the theoretical calculation presented, and about what are the authors actually checking when they compare their analytic expressions to the results obtained with another ''semi-analytical'' approximation.

As a result I have been drawn into comparing the results of this paper to earlier work in some detail to come to a reasonable conclusion with respect to what is the actual reach of this paper. I have found that the paper has many repetitions of earlier work and that it is extremely difficult to say what is truly new in the analytical calculation. The theoretical results are basically the same as those presented in ref [31], only that they are now given in the time domain and the form factor is evaluated with simple approximations of the shower model, a gaussian for the depth development and an exponential decay for the lateral distribution. Moreover the exercise of converting the frequency domain to the time domain is already the main content of ref [20], so perhaps the new contribution is to give an explicit expression in the time domain for the radio pulse away from the Cherenkov cone. [In ref [20] the explicit expression of the pulse in the time domain is only given in the Cherenkov cone. There is a small difference between the two expressions but no comment is made on this difference.  Is this a correction? Moreover reference [20] treats LPM showers incorrectly, because the frequency spectrum of the pulses has the
wrong low frequency limit (it should be the same as that of a shower of the same energy and no LPM effect). In this article the authors should discuss differences wrt ref [20] and address errors as well.]

The authors discuss in much detail the behavior of the results obtained in the time domain, relating them to shower properties. A non-expert reader could be misled into thinking these are new findings, but most of them are actually well documented in the abundant literature on this topic starting with the Monte Carlo simulation of ref [13] and practically in all the articles that have followed up to date. Stating these properties is not a problem of course, moreover it is a plus because they serve as a check giving confidence that the calculation gives reasonable results, however proper references should be made to articles that have actually found the same properties with similar or different techniques. Leaving aside the inadequacy of the presentation of the results (to be addressed below), the immediate question that crops up is, is this result of enough new content to justify a new article in PRD? The
answer in my opinion is that the result could be of interest for the field if the presentation of this is clear and the calculation leads to a compact expression (it appears to be so) which describes the pulses in this region with sufficient accuracy.

There is a phenomenological part of the article that compares these analytical results in the time domain to an alternative calculation. This comparison is used to find the fit parameters for the physical
scales of the showers. This, I believe, is new and maybe the authors are intending to justify the approximations made in the new analytical expression by making a detailed comparison of the result with a
reference calculation. Putting aside all the doubts that one can have about the procedure used for such a ''validation,'' the same question comes up, is this sufficient to be addressed in a PRD paper? In my
opinion the answer is that maybe it is enough, given the outstanding effort that is being made to exploit the radio technique in ice in the last decade, and the current interest that the technique has for
neutrino astronomy. It is true that simple models for radio pulses could be of much use at different levels in such experiments, as the authors state.  

Due to issues addressed above and a number of important issues that I will address below, it is clear to me that the article cannot be published in its current form and that major rewriting of the article is required for it to be published in any journal. This would imply adjusting the article to address my general comments above and to a number major issues that are very diverse and I will try to summarize
in the text to follow. These problems are of sufficient severity and number that the final decision on whether the article is adequate for being published in PRD would depend on how they are finally addressed
by the authors and I will not make a definitive statement.  \\

Major comments: \\

The length of the article. This article is far too long for the amount of information given. More importantly it is written in a cumbersome style so that it is very tedious to find out what is exactly being done. It should be drastically shortened and simplified.  

There is a lot of spurious text. Parts of the text are clearly to be completely removed. I was particularly surprised to find out that section III B, that takes nearly two pages and seems an elaborate
calculation, is simply an estimate of the width of the well known Greisen and Gaisser parameterizations, as obtained approximating them with gaussian functions. This is efficiently described in ref [31]
already (which is the basis of this work) with a reference to Bruno Rossi. Most people working with showers have approximated the shower development with gaussians so it seems too trivial to be reported in
detail. Moreover the derivation is particularly cumbersome. I do not even think it should be put in an appendix, the final result can just be displayed after a brief and clear description of what is actually
being done. Incidentally references could be made to [31] and to the work of Andringa,S. et al. Astrop Phys 34 (2011) 360 describing both the gaussian approximation, fluctuations in the longitudinal spread
(as named in [31], this is a better choice than ''longitudinal length'') and corrections to it. Incidentally the authors have payed very little attention to fluctuations in the shower profile which are relevant. It is particularly relevant to understand how the fluctuations in the longitudinal spread relate numerically to the small changes as the energy is varied. The sections on the uncertainty principle are in my opinion a distraction, it is not clear why they are introduced (nor is it clear that the principle is always satisfied, which is suspicious). 

A large number of variables is used together with several unfortunate choices of notation. Although I would usually classify such a comment under "minor issues about style", this article is particularly
exaggerated in this respect and more importantly it becomes an obstacle to clarity, obscuring the reasoning and the expressions presented and contributing to the confusion of the reader. Here are
some examples, again the list is not exhaustive: The authors use unnecessary changes of variables introducing names such as $p,q,u, x, x_{c}, \omega_{0}, \omega_{CF}, \omega_1$ ... that either giving little intuition to interpret the expressions, or I cannot see the need to introduce them all, and that in my opinion obscure the interpretation. In places the authors introduce redundant variables, for instance
there is a cutoff frequency $\omega_{CF}$ and a second one $\omega_0$ which is simply $\omega_0=sqrt(2/3) \omega_{CF}$. One could question the need to introduce two frequency scales which are so close, when a single one would do the job. The introduction of $\sigma = \omega/\omega_{CF}$ is another unnecessary new variable. For the lateral distribution a simple exponential is used as an approximation. The radius is labeled ``rho'' in cylindrical coordinates and a length scale is introduced as $1/\rho_0/\sqrt{2 \pi}$ which is extremely confusing because here $\rho_0$ is used as an inverse radial distance. $1/
\rho_0$ is related linearly to the Moliere radius (incidentally the definition given of the Moliere radius is incorrect). It is unnecessary to introduce two scales for the lateral spread given the simple approach given in this article, one of the two is enough to set the scale and using two adds to the confusion of variables, particularly with the unnatural definition of $\rho_0$. Another example of the latter type is introducing $u=1-i \eta$, which is not justified because afterwards an expansion is made in powers of $(1-u)=i \eta$, when $u$ tends to 1. Clearly $u$ is redundant and the natural variable to use is $\eta$, the limit corresponding to $\eta$ going to zero.  These problems include a few examples of bad choices of variables and labels, but the list is longer. The article should reduce the number of variables, change names where they lead to confusion such as the previous case, and eliminate spurious definitions that obscure the mathematical steps. Incidentally what I am demanding about improving the notation would end with a notation very close to that used in reference [31], which is an excellent article and has a rational labeling of variables. It should be used by the authors as a guide of style. Moreover, using the same conventions would make it a lot simpler to relate their work to the original work by Buniy and Ralston. 

More careful referencing is compulsory. The article has many issues with the references which are not
acceptable. These are actually the most striking alarms on a first read. The first one is part of the first general comment, the way the article references earlier work on theoretical calculations is
misleading, because the authors are not clear enough about what is new and what is not. The article should make a fair summary of what is being done in relation to what was done in the past, particularly in
ref [31], what ref [20] added to it and finally limit the first part of this article to the minimal amount of material that is needed to present the time expression that follows from the work of [20] and the
approximations made for the form factor. I will mention a few others but the list is not complete.
In several places an inadequate reference is given, for instance the authors cite Hanson et al [20] for the Greisen model, they cite again [20] for showers with many maxima because of the LPM effect but the
showers described in [20] have no multiple peaks, and again refer to [20] when they say that the lateral cascade width has the effect of a ``low-pass filter,'' while [20] may be the first use of such a "term",
such a reference is somewhat misleading in the sense that the effect of the cutoff in the frequency spectrum related to the lateral spread of the shower is described in multiple works on this topic, including the early ZHS MC calculation. The citation of the semi-analytic approach, ref [19], states that the method was ``introduced ... to account for non-Gaussian fluctuations in the charge excess profile.''
This is simply not true, the method was introduced to provide a fast and accurate method to describe radio pulses in general (similar to the motivation behind the analytic calculation the authors are discussing). The list again is longer. Such a careless citation style is not acceptable.

The expression of the pulse in the time domain away from the Cherenkov angle could be of interest. Yet the presentation of this result is again not clear. The authors have split the calculation into the text
and the appendix in a way that is not adequate because the text cannot be followed without reading the appendix. The expression involves several approximations, firstly an expansion for small $\eta=\omega/\omega_{CF}$ (more clear than in $(1-u)$). The result is an integral which is claimed to be related to the line-broadening function. The comparison involves the definition of $\omega_1=t_r/(2p)$, which depends on the retarded time, $t_r$. this is a variable and therefore it is not a number. So the condition for approximating the integral obtained by the symmetric approximation $\omega_1<\omega$ is clearly not satisfied for $p$ small, for $t_r$ large and obviously for low values of $\omega$ that are unavoidable in the $\omega$ integral from infinity to -infinity. The justification of this approximation is non-existent and deserves at least a careful discussion. 

I am doubtful that this is a good approximation. In any case the authors could argue that the expression obtained this way is a reasonable approximation to the pulse. In this respect the latter part of the article, which compares these results to a reference calculation, takes some relevance. I did not find such reasoning though and if this is the intention of the authors it should be clearly stated. Indeed when one looks at the comparisons made for the off-cone case it is apparent that the agreement is a lot worse than for the on-cone case. Given the number of approximations made, another question crops up. What are the advantages of this expression relative to those of ref [13] that are intended to provide a good and rapid description of the pulses? This should be addressed for completitude.  Moreover the approach used is clearly not valid for the LPM effect (this effect was also addressed in ref [20] with flaws), while the approach of [13] can deal with fluctuations of the shower profile and with the completely different profiles of showers fully affected by the LPM effect. This issue deserves a discussion.  

Incidentally a comparison is not made relative to a detailed simulation but rather to a calculation based on [13] that uses the simulation of the radio pulse in the Cherenkov cone to calculate the radio pulse away from it. This presumably has been done because [13] is fast but it is nevertheless an indirect way to compare. Some discussion of this is missing and multiple references are given to a simulation package that includes the approach of [13], what again leads to confusion because the reader may doubt whether they are comparing the parameterization to a MC simulation or to another model. If this is the case it needs to be discussed.

There are many other issues of the paper that I am not willing to spend more time in writing because the condition for me to accept this article for publication is that the authors should rewrite it according to all these recommendations and concerns. \\ \vspace{0.25cm}

Report of Referee A, DF12830/Hanson, September 12, 2021  \\ \vspace{0.25cm}
\hrulefill

The manuscript derives a fully analytic model of Askaryan radiation in
ice in the time domain. This manuscript will make an excellent
addition to the literature on modeling Askaryan signals in ice after
some revisions are made. I have five larger comments and then several
more minor ones.

\begin{enumerate}
\item Section V: Off-cone field equations is very hard to follow in part
because the derivation starts from an assumed form for
$\mathcal{\vec{E}}$ (line 4 in Table III) that is not presented earlier
in the paper, but also because there are several changes of variables
including reusing x, u and swapping their order. \\ \\
\textit{We now provide one appendix for the on-cone derivation, and one appendix for the off-cone derivation.  We tried to discern what details were important enough to keep in the main body of the text.  For the on-cone case, we felt that stating the problem, then introducing the inverse Fourier transform were important enough to keep in the main body.  For the inverse Fourier transform, we chose to explain the nature of the poles, and the two cases of the sign of retarded time in words in the main body.  We still conclude with the verification of the uncertainty principle for the on-cone case.  The rest of the details are now in Appendix B, along with the full definition of $\mathcal{E}$.  For the off-cone case, we re-introduce the main field equation, but rely on the Appendix A for the full specification of $\vec{\mathcal{E}}$.  We start the expansion and explain in words how to reach the line-broadening function (related to the Voigt function).  From there, we explain in words the \textit{symmetric approximation}, but relegate the details to Appendix C.  We then quote the result and verify the uncertainty principle.  Appendix B contains enough detail for the skeptical reader to follow the reasoning from start to finish.  Of course, we welcome any further suggestions from the reviewer.  We went a step further and included the form factor details in Appendix A as well.}
\item The comparison between NuRadioMC and the analytic model using digitized waveforms should be improved. There are several discussions of artifacts found in the analysis that would be eliminated if the authors just asked the waveforms directly from the authors of Ref. [23] or running NuRadioMC for example cases themselves. \\ \\
\textit{Our response to this general comment is why we have taken so long to respond.  We thank the reviewer for patience and understanding.  We decided to install and learn NuRadioMC tools, and to create scripts that produced output fields.  Further, since the goal was to compare MC truth values from NuRadioMC to values derived from field equations, we had to modify NuRadioMC to pass derived quantities like the longitudinal width parameter, $a$.  We began this effort right at the end of the summer during preparations for the Fall 2021 semester.  We are happy to report several major improvements to the waveform comparison analysis.  We chose to stay with a small number of explicit waveforms, hoping to keep the analysis repeatable for another user.  We find that the electromagnetic and hadronic on-cone waveforms from NuRadioMC (ARZ 2020 model) strongly agree with our on-cone formula.  In particular, the higher energy hadronic waveforms showed improvement in correlation coefficient and fractional power difference.  Further, electromagnetic fits to on-cone waveforms seem to produce agreement between the waveform $a$-value and the $a$-value from NuRadioMC.  The $a$-value agreement is not as good for the hadronic case, but still at the 10-50\% level.  As for the off-cone cases, both the electromagnetic and hadronic waveforms imporved in their correlation and power match with respect to NuRadioMC ARZ2020.  We find the correct viewing angle in all cases, and we even show that if we know $\theta$ in advance, the correlation between $a$-values from NuRadioMC and the waveforms is 97 percent.  This suggests an energy reconstruction that is robust to changes in comparison waveforms.  Finally, we find that our $\epsilon$ parameter appears to be greater than 1 for on-cone electromagnetic (10 PeV) events, while it is less than 1 for hadronic (100 PeV) events.  This is intriguing and requires further study.  We also highlight in several places where we intend to perform large-scale statistical analysis of our analytic model parameters in comparison to semi-analytic parameterizations.  We hope these improvements satisfy the reviewer's concerns.}
\item Given that the manuscript derives on- and off-cone equations, it seems appropriate to consider the limit of the off-cone equations as they approach $\theta_C$. What are the boundaries between using the on- and off-cone equations? \\ \\
\textit{We now provide further clarification surrounding the symmetric approximation, and connect that idea to a new subsection, VB.  Section VB is entitled Usage of the On-Cone versus Off-Cone Fields.  We note that taking the limit $\theta \to \theta_{\rm C}$ means $p \to 0$, and that the pulse amplitude (and therefore power) goes to zero for the off-cone case.  At first, it seems like the amplitude should diverge, but the pulse width goes to zero.  Conversely, as pulse width increases, amplitude decreases.  If one tries to make the amplitude of the off-cone model similar to the on-cone model, this fails because the pulse width goes to zero.  Observed power should be maximized at $\theta = \theta_{\rm C}$, and decrease with increasing $\Delta\theta$.  Thus, the pulse width of the off-cone field should be greater than or equal to the on-cone pulse width.  We do not prove that this is the only requirement one can impose, but it yields a physical result in the form of an equation for $\Delta\theta_{\rm min}$.  We also estimate the minimum appropriate angle to evaluate the off-cone field to be 1 degree off-cone. We hope this argument makes sense, and welcome any suggestions from the reviewer.}
\item This is really a stylistic comment, but it strikes me that several
aspects of the derivations are too detailed and could be removed or
moved into an appendix. I recommend the later to keep the main body of
the text shorter, while maintaining the authors’ preference for
presenting the complete derivation (which I appreciate). The advantage
of a shorter main text is that the primary results can come through
more cleanly. But again this is a style comment and I leave it to the
authors what their preference is. This could perhaps help with comment
\#1. \\ \\
\textit{We hope our changes discussed in response to main comment \#1 address this point as well.  We have added two appendices that contain all the details not in the main body.  We are always open to further suggestions.}
\item This treatment is valid under the assumption that $\eta < 1$, which
is mentioned at the beginning of the manuscript and used in the
derivation. The conclusions and summary should discuss the
implications of this assumption or whether there will be followup work
for $\eta > 1$. \\ \\
\textit{We have provided a paragraph at the end of Sec. VII A that describes the limitations imposed by requiring $\eta < 1$.  First, we review the several reasons $\eta < 1$ was necessary for the calculations.  Second, we estimate the minimum observer distance given $\eta = 1$ for typical parameters within $\eta$.  We find that the restriction is a mild one.  According to NuRadioMC, most UHE-$\nu$ Askaryan signals arrive from distances greater than the result implied by $\eta \leq 1$.  Third, we describe how future work will test the limits of $\eta > 1$.  Since this work already reveals a potential energy reconstruction, we must embark on a study to develop the probability distributions of parameters like $a$, $\omega_{\rm C}$, and $\omega_{\rm 0}$ when our model is forced to match the other models in NuRadioMC.  In these statistical comparisons it will be straightforward to determine in parallel how necessary it is to assume $\eta < 1$.  If the reviewer has any insight for future directions, we'd be happy to consider those as well.}
\end{enumerate}

Minor Comments by page:
\begin{itemize}
\item Page 1: Abstract: ``a simulation being used to design IceCube-Gen2” $\rightarrow$ ``a
simulation being used to design the radio component of IceCube-Gen2'' \\ \\
\textit{We adopted this change (see abstract).}
\item Page 1, Para 2: Clarify that the charge that radiates is the summed charge of the shower \\ \\
\textit{We adopted this change (see paragraph 2, page 1) by adding ``the net charge in the cascade''}
\item Page 1, Para 2: The discussion of the flux constraints strikes me as out of place. I recommend making a new paragraph prior to this one that describes the flux constraints across the energy band of interest to the radio component of Gen2.  Furthermore, why not use the IceCube and/or Auger constraints here which are more constraining than ANITA at EeV energies? Or perhaps add them together. \\ \\
\textit{We adopted this change, in that we introduce the IceCube EHE differential limit.  Instead of creating a new paragraph, we decided to allow the reasoning to ``flow down'' from the Askaryan effect, to the low UHE-$\nu$ flux, to the conclusion that large in situ detectors are being developed.}
\item Page 1, Para 4: What can be approximated as a Gaussian? The radiation or Xmax? \\ \\
\textit{Technically we meant the charge excess in the cascade.  We deemed this phrasing unnecessary, so we decided to cut it since it was confusing (see current paragraph 3).}
\item Page 2, Last Para Section I: Seems odd to compare waveforms to ``NuRadioMC'' rather than to the ZHS or ARVS models. In principle, the authors are using NuRadioMC but the comparison is to the specific semi-analytical model used in Ref. 23. \\ \\
\textit{We adopted this by changing ``by NuRadioMC'' to ``by the semi-analytic parameterizations in NuRadioMC.''}
\item Section II Para 1: ``index value'' $\rightarrow$ ``refractive index for solid ice'' \\ \\
\textit{We adopted this change verbatim.}
\item Page 4: Typo in this line (subject / verb agreement): ``As presented in Section II the form factor for the cascade ICD be the 3D Fourier transform of $f(z', \vec{\rho}')$.'' \\ \\
\textit{This is now fixed.}
\item Eqns 8-11 could be written more succinctly by just saying in words that the normalization factors are chosen such that integrating the form factor integrating over a sphere is normalized to 1. \\ \\
\textit{We adopted this change.  We now write in words that $f$ is normalized to 1.}
\item Notation is inconsistent in the format of the vector in the dot product ( $\vec{q} \cdot \vec{x’}$) going from Eqn (12) to (13). (Eqn 12 has arrows over the vectors while Eqn 13 typesets the vectors in bold.) \\ \\
\textit{This is now fixed.}
\item Paragraph between Eqns (15) and (16): Isn’t $\gamma$ the exponential scale factor of the ICD rather than the slope? \\ \\
\textit{We adotped the change.  Sometimes we use slope and scale factor interchangeably.  It now reads exponential scale factor.}
\item Page 5: Eqns 24 through 27 seem unnecessary. I could imagine that you could skip directly to Eqn (28) or move several aspects of these derivations into an appendix. But I suppose this is somewhat of a stylistic choice. \\ \\
\textit{We extra-adopted this one by creating Appendix A for the form factor.  We felt quoting the integral tables will give future readers the ability to check for themselves.}
\item Fig. 2: I recommend adding the original ZHS parameterization to this plot for comparison. \\ \\
\textit{If we adopted this edit, the ZHS curve would lie right on top of the gray curve of Fig. 2.  That is the reason we have (what is now) Eq. 11 (Eq. 31 in the first draft).  We did, however, discover a bug in the plotting script that creates Fig. 2.  The curves are now more aligned, as they should be.}
\item Page 6: Eqn (49) I would revise so that the sqrt is over the entire equation, but this is a style choice. It also also might be useful to point out that since $f(x)$ is a normalized function $<1$, $\ln(f)$ will always be negative and $a$ is real valued. \\ \\
\textit{We revised the format to keep the $\sqrt{\ln(x)}$ energy dependence by itself, but used the same symbol for both roots.  We added a sentence reminding the reader that $a$ will be real-valued.}
\item Page 7: Eqns (60) and (61) The approximations are valid if the approximations are valid if $\omega << \omega_C$ and $\omega << \omega_{CF}$ rather than only less than. \\ \\
\textit{Note that these formulas are now in Appendix B.  We chose to keep the language ``such that the following approximations of the factors in the denominator are valid.'' For example, the reader can see from Fig. 2 how close to 1 the parameter $\sigma$ can be.}
\item Page 8: For clarity, I recommend introducing the fact that you will change to the time domain between (63) and (64) \\ \\
\textit{We adopted this change by adding ``...converts the field to the time domain.''}
\item Page 10, Fig. 3: The differences in the curves are hard to discern, particularly at the peaks and the top and bottom curves are almost indistinguishable. Perhaps use different colors or grey scales? Or offset them in time? Explain why you chose these parameters as examples either in the text or in the caption. \\ \\ 
\textit{We adopted this edit by altering Figure 3 in the following ways: we now use gray scale for three cases, changed the normalization, explain in the text that the values are motivated by fit comparisons in Sec. 6, and the two plots are the same data.  The second plot is a zoomed in version of the first.  The data in the graphs are normalized to the black curve peak.}
\item Page 10: First paragraph: Is $\omega_C$ missing an n? \\ \\
\textit{Normally, yes, equations should have $n$'s.  In this work, in the units and vocabulary section, we state that we take things like $k$ and $c$ to be the wavenumber and speed ``in the medium.''  In this case, the $n$'s would just cancel, so we chose to leave it as is.}
\item Section V: 2nd paragraph: Where is $\mathcal{\vec{E}}( \omega, \theta)$ defined? If it's already derived in a reference, please cite it. \\ \\
\textit{Fixed: the full version of $\mathcal{E}$ is given in Appendix B.}
\item Eqn 85 and subsequent eqns in Section V: The equations are being evaluated at $u=1$, but $x \rightarrow 1$ in the argument of $\mathcal{\dot{E}}(u,x)$ rather than $u$. I would expect it to be written $\mathcal{\dot{E}}(u,1)$. \\ \\
\textit{Fixed.  The equations are now in Appendix B.}
\item Page 12, Para 1: Several variable (e.g. x) are reused multiple times in this manuscript. \\ \\
\textit{What's funny is if you recite the alphabet, we have used almost every letter at some point in the paper.  We might as well stay with x for the ratio of cascade energy to critical energy.  We think given the tables of definitions, it's clear what we mean in the different cases.}
\item Page 12, Para 1: Voigt function $\rightarrow$ Voigt function, $U(x,s)$. (I assume.) \\ \\
\textit{We adopted this change, though we recast it as the line-broadening function (the relevant case of the Voigt function).  We write ``the line-broadening function, H...'' and then use H subsequently.}
\item Page 13, Section VI: Para 1: While NuRadioMC includes semi-analytic models, they are all based on other work (eg. ZHS, AVRZ, etc) so you might as well cite the specific model that they used in Ref. 23 rather than the wrapper class. \\ \\
\textit{We adopted this change.  We cite the most recent ARZ result, as the waveforms are compared to that semi-analytic model using NuRadioMC.}
\item Eqn (108): $(\Delta f)^2$ seems to be a poor choice of variable name since it has units of power, but other variables in the fit are named ``f'' and represent frequencies. \\ \\
\textit{Fixed: we can change to capital E to avoid confusion, as in $(\Delta E)^2$}
\item Pages 13-19, Figs 6-9: The white space around the waveforms is too large and the black curve hides the peak of the gray curve. I suspect that some differences are hidden to the reader. \\ \\ 
\textit{We decided to zoom in for the x-axis to reduce whitespace.  For the y-axis, it's not that differences are being hidden, it's that the theoretical fit is minimizing $(\Delta E)^2$ and maximizing $\rho$.  Thus, the waveforms should lie on top of each other as much as possible.}
\item Tabs V-VI, IX: Where appropriate please add the Monte Carlo truth to these tables. I also don’t understand why ``PeV'' is listed in bold. \\ \\
\textit{We adopted this change by changing PeV to regular face instead of bold, and adding the new MC truth values for $a$ and $\theta$ to these tables when possible.}
\item Tab VII, IX: The description of the columns in the caption and the columns in the table are not consistent. \\ \\
\textit{Fixed, the captions are now consistent with the columns for the off-cone tables.}
\item Tab VIII: Why is waveform \#1 missing from this table? \\ \\
\textit{Fixed.  One waveform in the NuRadioMC paper for that EM energy is cut off visually, so we could not digitize it completely.  We wanted others to be able to repeat our analysis, so we digitized from the paper.  Now that we have transitioned to generating NuRadioMC waveforms, this is no longer an issue.}
\item Page 20, Section VII, Last Para: I agree that the potential for reconstruction is really intriguing. I would add a comment that this could be comprised of a study using the full statistical comparisons to the semi-analytical models. \\ \\
\textit{We adopted this change by adding a sentence that future work will build the probability distributions of parametes like $a$ and the cutoff frequencies.  The hope is to have an energy reconstruction independent of waveform normalization (in the y-axis).}
\end{itemize}

\end{document}